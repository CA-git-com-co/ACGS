% %%%%%%%%%%%%%%%%%%%%%%%%%%%%%%%%%%%%%%%%%
% %             Discussion                %
% %%%%%%%%%%%%%%%%%%%%%%%%%%%%%%%%%%%%%%%%%
\section{Discussion}\label{sec:discussion}
Our results demonstrate that \acgs{} provides a robust and practical solution to the evolutionary governance gap. The framework's core innovations---co-evolutionary adaptation, automated policy synthesis, and democratic oversight---collectively enable a new form of intrinsic, adaptive AI governance.

\subsection{Theoretical and Practical Implications}
Theoretically, our work provides the first formal model of co-evolutionary governance with proven stability guarantees. The successful empirical validation of the Lipschitz constant and convergence rate bridges the gap between AI governance theory and practice. Practically, \acgs{} offers a concrete architectural blueprint for building trustworthy AI systems. By embedding governance directly into the operational loop, the framework moves beyond reactive, external oversight to proactive, ``compliance-by-design'' enforcement. The production deployment on Solana validates its applicability to high-stakes, decentralized environments~\cite{solana2020, quantumagi2024}.

\subsection{Limitations and Sociotechnical Challenges}
Despite the encouraging empirical results, four key limitations warrant continued investigation:

\begin{itemize}[leftmargin=*, itemsep=2pt]
  \item \textbf{Distance-Metric Validity}\quad Our Lipschitz stability proof depends on semantic and behavioral distance metrics that remain an open research area.  Although we empirically achieved $L<1$ across multiple embeddings, further work is needed to validate metric robustness at higher constitutional complexity.

  \item \textbf{Democracy at Scale}\quad The Constitutional Council’s anti-capture design improves upon prior participatory failures, yet scaling deliberative democracy to global AI governance---while preserving procedural legitimacy---demands sociotechnical research beyond our technical scope.

  \item \textbf{Reliability of LLM+SMT Synthesis}\quad Our pipeline attains 99.7\percent{} post-validation accuracy, but the completeness of formal verification for nuanced principles and the brittleness of LLM policy generation both merit deeper study.

  \item \textbf{Quantum-Model Verification}\quad The QPE superposition layer introduces additional complexity.  Formal guarantees of deterministic collapse and entanglement integrity require expanded verification frameworks and threat modelling against novel attack surfaces.
\end{itemize}
