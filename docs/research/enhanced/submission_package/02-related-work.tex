% %%%%%%%%%%%%%%%%%%%%%%%%%%%%%%%%%%%%%%%%%
% %             Related Work              %
% %%%%%%%%%%%%%%%%%%%%%%%%%%%%%%%%%%%%%%%%%
\section{Related Work}\label{sec:related_work}
Our work builds on three pillars of modern AI safety and governance research: policy-as-code frameworks for technical operationalization, constitutional AI approaches for value alignment, and runtime enforcement systems for agent safety. \acgs{}\ synthesizes these domains to address a critical gap: the transition from static governance mechanisms to dynamic, co-evolutionary systems that adapt alongside the AI systems they govern.

\subsection{Democratic Oversight}
High-level AI governance frameworks, such as the NIST AI Risk Management Framework~\cite{nist2023ai} and ISO/IEC~42001~\cite{iso42001}, provide essential organizational guidance but lack mechanisms for technical operationalization. Policy-as-Code (PaC) paradigms, exemplified by Open Policy Agent (OPA)~\cite{opa2023} and its policy language Rego~\cite{rego2019}, bridge part of this gap by enabling executable governance rules. However, PaC systems traditionally rely on manually authored rules, creating a bottleneck that inhibits rapid adaptation to evolving AI behaviors. \acgs{}\ advances this foundation by automating rule generation from high-level constitutional principles, enabling governance systems to evolve at machine speed rather than human timescales.

\subsection{Adversarial Robustness}
Anthropic's Constitutional AI (CAI) represents the current state-of-the-art in principle-guided AI behavior, successfully achieving 95\percent{} jailbreak prevention through Constitutional Classifiers and engaging approximately 1,000 participants in democratic principle selection via Collective Constitutional AI~\cite{anthropic2022constitutional,anthropic2023collective}. However, CAI operates exclusively during training time, embedding principles statically into model weights. This creates what recent analysis identifies as ``normative thinness''---the inability to adapt constitutional principles to post-deployment challenges without complete retraining~\cite{digitalconstitutionalist2023}.

Recent incidents demonstrate this limitation: ChatGPT's training-time safety measures proved inadequate against novel jailbreaking techniques that emerged months after deployment, requiring reactive patches rather than constitutional adaptation~\cite{wei2023jailbroken}.

\acgs{}\ addresses this fundamental limitation by implementing the first runtime constitutional system where principles evolve dynamically while maintaining democratic legitimacy, representing a paradigmatic shift from constitutional embedding to constitutional evolution.

\subsection{LLM Synthesis}
Recent work demonstrates the potential of LLMs to translate natural language into structured code and policies~\cite{propertygpt2023, veriplan2023}. However, challenges such as semantic inaccuracy and hallucination persist. We address these through a multi-stage validation pipeline that integrates syntactic checks, semantic alignment scoring, formal verification for critical rules, and human-in-the-loop oversight.

\subsection{Real-Time Enforcement}
Frameworks like AgentSpec~\cite{agentspec2023} and Progent~\cite{progent2023} provide runtime safety constraints for LLM~agents, demonstrating the feasibility of real-time governance enforcement. These systems excel at preventing harmful behaviors but depend on static, manually crafted rule sets that cannot adapt to novel scenarios or emergent behaviors. The PGC component of \acgs{}\ draws inspiration from these runtime guards but uniquely integrates enforcement with an adaptive, constitutionally-grounded rule synthesis engine, enabling governance that responds to new challenges while maintaining constitutional consistency.

\subsection{Governance of Evolutionary Computation}
The governance of EC systems is a nascent field. While some research explores synergies between LLMs and EC, it typically focuses on using LLMs to enhance evolutionary operators. \acgs{}\ is distinct in establishing a co-evolutionary loop where the governance framework itself evolves in response to the emergent behaviors of the EC system, creating a system of checks and balances that adapts at machine speed.
