% %%%%%%%%%%%%%%%%%%%%%%%%%%%%%%%%%%%%%%%%%
% %              Introduction             %
% %%%%%%%%%%%%%%%%%%%%%%%%%%%%%%%%%%%%%%%%%
\section{Introduction}\label{sec:introduction}

The rapid deployment of AI systems across critical domains has exposed a fundamental limitation in current governance approaches: the inability to adapt constitutional principles at machine speed to address post-deployment challenges. Recent high-profile incidents illustrate this \textit{constitutional adaptation gap}. In 2023, ChatGPT's training-time safety measures proved inadequate when faced with novel jailbreaking techniques that emerged months after deployment~\cite{wei2023jailbroken}. Similarly, GitHub Copilot's code generation system, despite extensive pre-training safety protocols, required multiple post-deployment interventions to address copyright violations and security vulnerabilities that emerged through user interaction patterns not anticipated during training~\cite{copilot2023incidents}.

These failures highlight a critical limitation in current Constitutional AI approaches: while Anthropic's Constitutional AI~\cite{anthropic2022constitutional} successfully embeds principles during training, achieving 95\percent{} jailbreak prevention through Constitutional Classifiers, it cannot evolve these principles in response to emerging regulations, novel attack vectors, or changing stakeholder values without complete retraining. This creates critical windows of vulnerability where AI systems operate outside intended governance boundaries, particularly in evolutionary computation (EC) systems that continuously modify their own behavior through population dynamics, mutation, and selection processes~\cite{russell2020artificial}.

Consider an EC system optimizing a supply chain: without governance, it might evolve solutions that achieve efficiency by subtly violating fairness constraints (e.g., discriminating against certain suppliers) or exploiting security vulnerabilities that human designers never anticipated. Static governance rules cannot adapt to such emergent behaviors, while human oversight operates too slowly to intervene in real-time evolutionary processes.

We introduce \acgs{}, the first co-evolutionary constitutional governance framework that enables constitutional adaptation at machine speed while maintaining democratic legitimacy. Unlike training-time approaches that embed static principles, \acgs{} implements dynamic constitutional evolution through runtime governance that adapts to emerging challenges without requiring system retraining. This represents a paradigmatic shift from \textit{constitutional embedding} (training-time principle fixation) to \textit{constitutional evolution} (runtime principle adaptation), addressing what recent analysis identifies as the ``normative thinness'' of current Constitutional AI approaches~\cite{digitalconstitutionalist2023}.

At its core, \acgs{} uses Large Language Models (LLMs) to dynamically synthesize a living constitution, encoded as executable policies in Rego~\cite{rego2019}, and enforces them in real-time via a Prompt Governance Compiler (PGC) based on Open Policy Agent (OPA)~\cite{opa2023}. The framework's architecture enables governance mechanisms and AI systems to co-evolve, facilitating ``constitutionally bounded innovation'' that maintains safety and compliance while preserving system adaptability.

This work makes three key contributions to AI governance and EC\@:
\begin{enumerate}[leftmargin=*,topsep=2pt,itemsep=2pt,parsep=0pt]
    \item \textbf{Co-Evolutionary Governance Architecture:} We introduce a governance framework that evolves alongside the AI system it governs, addressing the mismatch between static governance and dynamic AI behavior through a four-layer architecture with democratic oversight mechanisms, including multi-stakeholder Constitutional Councils, cryptographically secured amendment procedures, and transparent appeal workflows.
    \item \textbf{LLM-Driven Policy Synthesis and Enforcement Pipeline:} We develop an end-to-end mechanism for translating natural language principles into executable Rego policies with real-time enforcement, achieving 73--93\percent{} synthesis success through multi-tier validation and demonstrating sub-50\ms{} policy enforcement (42.3\ms{} average) suitable for integration into evolutionary loops.
    \item \textbf{Formal Stability Guarantees and Empirical Validation:} We provide theoretical foundations with formal stability proofs (Lipschitz constant $L = 0.74 < 1$) and comprehensive empirical validation demonstrating constitutional compliance improvements from 31.7\percent{} to 94.7\percent{} in EC systems, with full implementation and reproducible artifacts released for open science (see Appendix~\ref{sec:appendix_artifacts}).
\end{enumerate}
