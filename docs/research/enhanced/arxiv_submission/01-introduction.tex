% %%%%%%%%%%%%%%%%%%%%%%%%%%%%%%%%%%%%%%%%%
% %              Introduction             %
% %%%%%%%%%%%%%%%%%%%%%%%%%%%%%%%%%%%%%%%%%
\section{Introduction}\label{sec:introduction}

Evolutionary computation (EC) systems represent a critical frontier in AI safety research, where traditional governance approaches fundamentally break down~\cite{amodei2016concrete,russell2019human}. Unlike deterministic AI systems, EC generates emergent behaviors through population dynamics, mutation, and selection processes that cannot be predicted or controlled by static rule sets~\cite{russell2020artificial}. This creates what we term the \textit{evolutionary governance gap}: the inability of existing AI governance frameworks---from regulatory standards like the EU AI Act~\cite{eu2024ai} to technical solutions like Constitutional AI (CAI)~\cite{anthropic2022constitutional}---to manage systems that continuously evolve their own behavior. This gap poses significant risks, as unconstrained evolutionary systems can develop unsafe, biased, or non-compliant solutions in society-critical applications~\cite{barocas2019fairness,acgs2024}.

Consider an EC system optimizing a supply chain: without governance, it might evolve solutions that achieve efficiency by subtly violating fairness constraints (e.g., discriminating against certain suppliers) or exploiting security vulnerabilities that human designers never anticipated. Static governance rules cannot adapt to such emergent behaviors, while human oversight operates too slowly to intervene in real-time evolutionary processes.

To bridge this gap, we introduce \acgs{}, a co-evolutionary constitutional governance framework that embeds adaptive principles directly into EC~systems. The framework's architecture enables governance mechanisms and the AI~system to adapt together, facilitating ``constitutionally\allowbreak\ bounded innovation''. At its core, \acgs{} uses Large Language Models (LLMs) to dynamically synthesize a living constitution, encoded as executable policies in Rego~\cite{rego2019}, and enforces them in real-time via a Prompt Governance Compiler (PGC) based on Open Policy Agent (OPA)~\cite{opa2023}.

This work makes three key contributions to AI governance and EC\@:
\begin{enumerate}[leftmargin=*,topsep=2pt,itemsep=2pt,parsep=0pt]
    \item \textbf{Co-Evolutionary Governance Architecture:} We introduce a governance framework that evolves alongside the AI system it governs, addressing the mismatch between static governance and dynamic AI behavior through a four-layer architecture with democratic oversight mechanisms, including multi-stakeholder Constitutional Councils, cryptographically secured amendment procedures, and transparent appeal workflows.
    \item \textbf{LLM-Driven Policy Synthesis and Enforcement Pipeline:} We develop an end-to-end mechanism for translating natural language principles into executable Rego policies with real-time enforcement, achieving 73--93\percent{} synthesis success through multi-tier validation and demonstrating sub-50\ms{} policy enforcement (42.3\ms{} average) suitable for integration into evolutionary loops.
    \item \textbf{Formal Stability Guarantees and Empirical Validation:} We provide theoretical foundations with formal stability proofs (Lipschitz constant $L = 0.74 < 1$) and comprehensive empirical validation demonstrating constitutional compliance improvements from 31.7\percent{} to 94.7\percent{} in EC systems, with full implementation and reproducible artifacts released for open science (see Appendix~\ref{sec:appendix_artifacts}).
\end{enumerate}
