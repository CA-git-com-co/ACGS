\usepackage{graphicx}
% \usepackage{subfigure}
\usepackage{natbib}
\usepackage{comment}

\usepackage{xcolor}
\usepackage{colortbl}
\usepackage{hyperref}
\definecolor{DarkBlue}{rgb}{0.1,0.1,0.5}
\definecolor{DarkGreen}{rgb}{0.1,0.5,0.1}
\definecolor{deepyellow}{RGB}{218, 174, 42}
\hypersetup{
	colorlinks   = true,
	linkcolor    = DarkBlue, % color of internal links
	urlcolor     = DarkBlue, % color of external links
	citecolor    = DarkGreen % color of links to bibliography
}


% \usepackage[pdftex]{graphicx}
\usepackage{xfrac}
\usepackage{amsthm}
\usepackage{thmtools,thm-restate}
\usepackage{amsmath,amsfonts,dsfont}
\usepackage[mathscr]{euscript}
\usepackage{booktabs} % for professional tables

\usepackage{multirow}
\usepackage{svg}
% \usepackage{subcaption}
\usepackage{makecell}
\usepackage{pgfplots}
\pgfplotsset{compat=1.18} % adjust to the version you have installed


\definecolor{DarkBlue}{rgb}{0.3,0.3,0.70}
\definecolor{azure}{rgb}{0.0, 0.5, 1.0}
\definecolor{darkcerulean}{rgb}{0.03, 0.27, 0.49}
\definecolor{denim}{rgb}{0.08, 0.38, 0.74}
\definecolor{DarkGreen}{rgb}{0.3,0.7,0.3}
\definecolor{lighter-gray}{gray}{0.95}
\hypersetup{
	colorlinks   = true,
	linkcolor    = DarkBlue, % color of internal links
	urlcolor     = darkcerulean, % color of external links
	citecolor    = denim % color of links to bibliography
}
% \definecolor{AlgHighlight}{rgb}{0.2,0.4,0.2}
\definecolor{AlgHighlight}{HTML}{228B22}

\newtheoremstyle{thmstyle}
{0.5em} % Space above
{0.15em} % Space below
{} % Body font
{} % Indent amount
{\bfseries} % Theorem head font
{.} % Punctuation after theorem head
{.5em} % Space after theorem head
{} % Theorem head spec (can be left empty, meaning `normal')

\theoremstyle{thmstyle} 
\newtheorem{theorem}{Theorem}
\newtheorem{lemma}{Lemma}
\newtheorem{proposition}{Proposition}[section]
\newtheorem{claim}{Claim}[section]
\newtheorem{corollary}{Corollary}
\newtheorem*{nonumtheorem}{Theorem}
% \newtheorem*{KL}{Klein’s Lemma}

\newtheorem{obs}{Observation}[section]

\theoremstyle{definition}
\newtheorem{definition}{Definition}
\newtheorem{assumption}{Assumption}
\newtheorem{exmp}{Example}[section]


\theoremstyle{remark}
\newtheorem{remark}{Remark}
\newtheorem*{note}{Note}
\newtheorem{case}{Case}

\newenvironment{proof_sketch}
    {\par\noindent\textbf{Proof Sketch.}\space}
    {\hfill$\square$\par\vspace{1ex}}

% \DeclareMathOperator*{\argmax}{arg\,max}
% \DeclareMathOperator*{\argmin}{arg\,min}

\DeclareMathOperator*{\argminmax}{arg\,min\,max}
\DeclareMathOperator*{\argmaxmin}{arg\,max\,min}


\newcommand{\gv}[1]{\ensuremath{\mbox{\boldmath$ #1 $}}} 
\newcommand{\grad}[1]{\gv{\nabla} #1} % for gradient
\newcommand{\divergence}[1]{\gv{\nabla} \cdot #1} % for divergence
\newcommand{\curl}[1]{\gv{\nabla} \times #1} % for curl
\definecolor{PaperGreen}{RGB}{0,150,0}   % strong green, prints well
\definecolor{PaperRed}{RGB}{100,0,0}     % deep red, good contrast on white

\newcommand{\changeUp}[1]{\textcolor{PaperGreen}{\textbf{#1\(\uparrow\)}}}
\newcommand{\changeDown}[1]{\textcolor{PaperRed}{\textbf{#1\(\downarrow\)}}}

%\setlength{\parindent}{0em}
%\setlength{\parskip}{1em}
%\renewcommand{\baselinestretch}{1.25}

\renewcommand{\1}{ \mathds{1}}
\newcommand{\A}{\mathcal{A}}
\renewcommand{\E}{\mathbb{E}}
\newcommand{\F}{\mathcal{F}}
\newcommand{\G}{\mathcal{G}}
\renewcommand{\H}{\mathcal{H}}
\newcommand{\I}{\mathcal{I}}
\newcommand{\J}{\mathcal{J}}
\newcommand{\K}{\mathcal{K}}
\renewcommand{\L}{\mathcal{L}}
\newcommand{\M}{\mathcal{M}}
\newcommand{\N}{\mathbb{N}}
\newcommand{\Ncal}{\mathcal{N}}
\renewcommand{\O}{\mathcal{O}}
\renewcommand{\P}{\mathcal{P}}
\newcommand{\p}{\mathcal{\pi}}
\newcommand{\prob}{\mathbb{P}}
\newcommand{\Q}{\mathcal{Q}}
\renewcommand{\R}{\mathbb{R}}
\renewcommand{\S}{\mathcal{S}}
\newcommand{\T}{\mathcal{T}}
\newcommand{\U}{\mathcal{U}}
\newcommand{\X}{\mathcal{X}}
\newcommand{\Y}{\mathcal{Y}}

\newcommand{\Regret}{{\rm Regret}}


\renewcommand{\b}{\beta}
\renewcommand{\l}{\lambda}
\newcommand{\x}{\mathit{x}}
\newcommand{\0}{\underline{\mathbf{0}}}
\renewcommand{\emptyset}{\varnothing}
\newcommand{\<}{\langle}
\renewcommand{\>}{\rangle}



\renewcommand{\I}{\mathcal{A}}
\renewcommand{\S}{\mathcal{C}}
\newcommand{\C}{\mathcal{C}}

%\renewcommand{\and}{\wedge}
%\renewcommand{\or}{\vee}



\newcommand{\tr}{\top}

\newcommand{\norm}[1]{\left\lVert#1\right\rVert}

%\newcommand{\midsepremove}{\aboverulesep = 0mm \belowrulesep = 0mm}
%\newcommand{\midsepdefault}{\aboverulesep = 0.605mm \belowrulesep = 0.984mm}

\newcommand{\CommentLines}[1]{}

\usepackage{array}
\newcolumntype{x}[1]{>{\centering\let\newline\\\arraybackslash\hspace{0pt}}m{#1}}
% \usepackage{subfigure}
\usepackage{colortbl}

% \usepackage{algorithm}
% \usepackage[ruled,vlined]{algorithm2e}
% \usepackage{algorithmicx}
\usepackage{wrapfig}
\usepackage{svg}


\usepackage{amsmath} % for checkmark
\usepackage{amssymb} % for checkmark
% \usepackage{geometry} % for better table width management
\usepackage{colortbl} % for cell colors
\usepackage{enumitem}


% Define colors for the heatmap
\definecolor{green}{HTML}{C6EFCE}
\definecolor{red}{HTML}{FFC7CE}
\definecolor{yellow}{HTML}{FFEB9C}
\definecolor{LightGray}{gray}{0.95}
\newcolumntype{H}{>{\columncolor{LightGray}}c}


\usepackage{listings}

\definecolor{darkgreen}{rgb}{0, 0.6, 0}

\lstset{
    basicstyle=\ttfamily\footnotesize, % Font style
    keywordstyle=\color{blue}\bfseries, % Keywords
    commentstyle=\color{darkgreen}, % Comments
    stringstyle=\color{red}, % Strings
    numberstyle=\tiny\color{gray}, % Line number style
    frame=lines, % Add a frame around the code
    breaklines=true, % Break long lines
    tabsize=4, % Tab width
    showstringspaces=false % Don't show spaces in strings
}

\usepackage{algorithm}
\usepackage{algorithmic}
% \usepackage{minted}
% \newcommand\Mycomb[2][^n]{\prescript{#1\mkern-0.5mu}{}C_{#2}}



\newcommand{\mpo}{\textsc{Mpo}}
\newcommand{\carma}{\textsc{Crome}}
\newcommand{\carmo}{\textsc{Carmo}}
\newcommand{\rrm}{\textsc{RRM}}
\newcommand{\odin}{\textsc{Odin}}
\newcommand{\rewordbench}{\textsc{ReWordBench}}

\newcommand{\independent}{\perp\!\!\!\perp} % Conditional independence symbol
\renewcommand{\rmdefault}{ptm} % Example using Times Roman font if desired

\newcommand{\gemmait}[1]{\texttt{Gemma-2-#1B-IT}}
\newcommand{\gemmathreeit}[1]{\texttt{Gemma-3-#1B-IT}}
\newcommand{\gemma}[1]{\texttt{Gemma-2-#1B}}
\newcommand{\qwen}{\texttt{Qwen2.5-7B}}

\definecolor{remarkblue}{HTML}{0B5CA3} 

\usepackage{listings}
% \usepackage[most]{tcolorbox}
% \remarkbox  ➜  usage:  \begin{remarkbox}{<your title>} … \end{remarkbox}
\newtcolorbox{remarkbox}[1]{%
  enhanced, breakable,
  colback=white,                  % body background
  colframe=remarkblue,            % border colour
  colbacktitle=remarkblue,        % title‐bar background
  coltitle=white,                 % title‐bar text colour
  boxrule=0.7pt,
  arc=2mm,                        % slightly rounded
  left=3mm,right=3mm,top=2mm,bottom=2mm,  % inner padding
  fonttitle=\bfseries,            % bold title
  title={\strut #1},              % \strut keeps the bar height uniform
  before upper=\itshape,          % body text italic
  after skip=1ex,                 % small space below the box
}

\definecolor{takeawayframe}{RGB}{45,82,160}   % frame colour
\definecolor{takeawayback} {RGB}{245,248,252} % background

%--------------------------------------------------------------------
% \takeawaybox[<key=value, ...>]
%--------------------------------------------------------------------
\newtcolorbox{takeawaybox}[1][]{%
  enhanced, breakable,
  sharp corners=south,         % same look you used before
  colback   = gray!5!white,
  colframe  = gray!10!white,
  boxrule   = 0.7pt,
  left      = 1mm,             % inner padding (defaults)
  right     = 1mm,
  top       = 0.5mm,
  bottom    = 0.5mm,
  before skip=0pt,             % no extra vertical glue
  after skip =0pt,             % ← behaves like a normal text line
  before upper=\text,       % italic body
  #1}                          % allow per-box overrides

