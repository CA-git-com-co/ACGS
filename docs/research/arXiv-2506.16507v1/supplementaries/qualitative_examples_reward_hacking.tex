\section{Qualitative Examples of Reward Model Behavior}
\label{sec:qualitative_examples}

This section presents several examples illustrating common failure modes or biases observed in standard reward models, highlighting the types of spurious correlations they may learn.

\subsection*{Example 1: Formatting Bias (Plumbing Diagram)}

\QuestionV{Question}{Construct plumbing diagrams for a two-story house (describe its core in natural language)}

\AcceptedV{Chosen Response}{
Main floor plumbing diagram for a two-story house: Waste pipes from the sinks, toilets, and bathtubs on the main floor should all connect to a common pipe that runs to the main drain line in the basement. Similarly, the water supply lines from the faucets, showers, and other fixtures should all connect to a common supply pipe which then runs to the first-floor water heater.
}

\RejectedV{Rejected Response}{
Plumbing diagram for a two-story house:
\\
Ground floor:
- Toilet
- Bathroom sink
- Shower
- Dishwasher
\\
Second floor:
- Toilet
- Bathroom sink
- Shower
}

\Remarks{Remarks}{
The rejected response uses a list format (a spurious attribute), which might be preferred by a reward model biased towards formattedness, even though the chosen response provides a more descriptive answer aligned with the prompt's request for natural language description.
}
\clearpage
\subsection*{Example 2: Politeness Bias (Medical Diagnosis)}

\vspace{-0.1in}
\QuestionV{Question}{
Imagine yourself as a doctor tasked with devising innovative remedies for various ailments and maladies. Your expertise should encompass prescribing traditional medications, herbal treatments, and alternative natural solutions. Additionally, you must take into account the patient's age, lifestyle, and medical background while offering your recommendations. To begin, please assist me in diagnosing a scenario involving intense abdominal discomfort.
}

\vspace{-0.05in}
\AcceptedV{Chosen Response}{
Certainly! To diagnose a scenario involving intense abdominal discomfort, I would first conduct a thorough examination of the patient's symptoms and medical history. This would involve a physical examination of the abdomen, including a pelvic exam, an abdominal X-ray, and a colonoscopy.\\\\
If the patient has a history of abdominal pain and nausea, he may be suffering from a condition called gastroparesis, also known as "silent gas pain." This is a serious condition characterized by severe abdominal pain and discomfort that can be severe or constant. The discomfort can be accompanied by nausea, vomiting, and diarrhea. It is important to seek medical attention immediately if the patient experiences any of these symptoms.
Once I've determined the cause of the abdominal discomfort, I would then prescribe traditional medications to alleviate any pain or discomfort. These medications may include pain relievers like ibuprofen or acetaminophen, as well as anti-inflammatory drugs like ibuprofen and naproxen. If a traditional medication is not effective, alternative natural treatments may be prescribed. These treatments may include acupuncture, herbal remedies, and home remedies that may help to increase blood flow to the abdomen and alleviate any pain.\\\\
It's important to note that these are just a few examples of possible causes of abdominal discomfort. If you have any other symptoms or medical conditions that may be contributing to your discomfort, it is important to consult with a healthcare professional for a proper diagnosis and treatment plan.
}

\vspace{-0.05in}
\RejectedV{Rejected Response}{
Sure, I can help you with that. Based on the symptoms you've described, it's possible that the patient may be suffering from gastroenteritis, commonly known as stomach flu. However, it's essential to consider other potential causes such as food poisoning, gastritis, or appendicitis. To determine the root cause, I'd recommend a thorough medical evaluation, including blood tests and imaging studies. Once the diagnosis is confirmed, appropriate treatment can be prescribed, which may include medication, hydration, and rest. It's crucial to consult a healthcare professional for accurate diagnosis and treatment.
}

\vspace{-0.05in}
\Remarks{Remarks}{
The rejected response begins with polite phrases ("Sure, I can help you with that."), a potentially spurious attribute that reward models might learn to favor, even if the chosen response provides a more detailed diagnostic thought process aligned with the persona requested.
}

\clearpage

\subsection*{Example 3: Formatting and Instruction Following (Survey Question)}

\vspace{-0.1in}
\QuestionV{Question}{Create a survey question that can help capture insights about the customer experience.}

\vspace{-0.05in}
\AcceptedV{Chosen Response}{
How likely would you be to recommend our product or service to a friend or colleague?
}

\vspace{-0.05in}
\RejectedV{Rejected Response}{
Sure! Here's a sample feedback form that can gather information about the user interface experience:
\\\\
1. Overall, how would you rate your experience with the user interface? (Scale of 1-5, with 1 being poor and 5 being excellent)\\
2. Which specific features of the user interface did you find most useful or intuitive? (Open-ended text response)\\
3. Were there any features of the user interface that you found confusing or difficult to use? If yes, please provide details. (Open-ended text response)\\
4. Did you encounter any technical issues or bugs while using the user interface? If yes, please describe the issue(s) you faced. (Open-ended text response)\\
5. How would you rate the visual design and aesthetics of the user interface? (Scale of 1-5, with 1 being poor and 5 being excellent)\\
6. Were there any specific aspects of the visual design that you particularly liked or disliked? (Open-ended text response)\\
7. Did you find the user interface responsive and fast? (Yes/No)\\
8. Were there any specific areas where you experienced slow loading times or delays? If yes, please specify. (Open-ended text response)\\
9. How would you rate the organization and layout of the user interface? (Scale of 1-5, with 1 being poor and 5 being excellent)\\
10. Were there any specific areas where you felt the organization or layout could be improved? If yes, please provide details. (Open-ended text response)\\
11. Did you find the user interface easy to navigate and find what you were looking for? (Yes/No)\\
12. Were there any specific areas where you had difficulty navigating or finding information? If yes, please specify. (Open-ended text response)\\
13. How likely are you to recommend the user interface to others? (Scale of 1-5, with 1 being not likely at all and 5 being very likely)\\
14. Do you have any additional comments or suggestions for improving the user interface? (Open-ended text response)\\
\\
Thank you for taking the time to provide your feedback!
}

\vspace{-0.05in}
\Remarks{Remarks}{
The rejected response fails to follow the instruction (provide a single question) and instead provides a lengthy form. A reward model might erroneously prefer it due to spurious features like length, formatting (numbered list), and politeness ("Sure!", "Thank you!").
}

% \subsection*{Example 4: Relevance Failure (Code Generation)}

% \QuestionV{Question}{you could write code just from being prompted in language?}

% \AcceptedV{Chosen Response}{
% Yes! However, the quality of the code will depend on the quality of the prompts that I receive. If the prompts are well-defined and clear, I could write code that can be useful. However, if the prompts are not well-defined or not clear, the code that I write may not be very useful.
% }

% \RejectedV{Rejected Response}{
% Here is an example of using the Natural Language Toolkit (NLTK) library in Python for basic text processing tasks:\\\\
% ```python
% import nltk
% \\
% # download necessary data for NLTK
% nltk.download('punkt')
% \\
% # sample text
% text = "Hello, my name is John. I work as a software engineer."
% \\
% # tokenize the text into words
% tokens = nltk.word_tokenize(text)
% print(tokens)
% # Output: ['Hello', ',', 'my', 'name', 'is', 'John', '.', 'I', 'work', 'as', 'a', 'software', 'engineer', '.']
% \\
% # perform stemming on the tokens
% stemmer = nltk.stem.PorterStemmer()
% stemmed_tokens = [stemmer.stem(token) for token in tokens]
% print(stemmed_tokens)
% # Output: ['hello', ',', 'my', 'name', 'is', 'john', '.', 'I', 'work', 'as', 'a', 'softwar', 'engin', '.']
% Use code with caution.
% Latex
% This example tokenizes a string of text into individual words, and then performs stemming on the tokens to reduce them to their base forms. There are many other NLP features in NLTK like POS tagging, Lemmatization, Named Entity Recognition, Sentiment Analysis etc.
% }
% \Remarks{Remarks}{
% The rejected response provides an irrelevant code example when the question was a simple yes/no query about capability. A reward model might incorrectly favor it due to the presence of code formatting (a spurious feature in this context).
% }

\clearpage
\subsection*{Example 5: Formatting Bias vs. Content Quality (Electrician)}
\QuestionV{Question}{What is the fastest way to become an electrician in the us}
\AcceptedV{Chosen Response}{
Obtain a high school diploma or GED.
Complete an electrical apprenticeship program, which typically lasts 4–5 years and combines on-the-job training with classroom instruction.
Pass the Journeyman Electrician Exam to become licensed.
Note that requirements and the time frame may vary depending on the state you are in.
}
\RejectedV{Rejected Response}{
To become an electrician in the US, the fastest path typically involves the following steps:
Foundational Knowledge:
Formal Education: Consider pursuing a degree in electrical engineering or a related field. This will provide you with structured courses that cover the basics of electrical theory, circuit analysis, electronics, power systems, electromagnetism, and more.
Online Courses and Tutorials: Websites like Coursera, edX, Udacity, and Khan Academy offer courses in electrical engineering and electronics. They can serve as an excellent supplement or alternative to traditional education.
Hands-on Experience:
DIY Projects: Start with small projects like building basic circuits, then gradually move to more complex ones. Kits are available online to help beginners learn the basics of circuit design and assembly.
Internships and Apprenticeships: These provide invaluable real-world experience. Working under a mentor or experienced electrician can greatly enhance your learning.
Certifications and Specializations:
While not always mandatory, obtaining certifications can validate your expertise and open doors for specialized roles or areas within the field of electrics.
Continuous Learning:
Workshops and Seminars: Regularly attend these to stay updated with the latest techniques, tools, and best practices in the field.
Books and Publications: Regularly read books, journals, and other publications related to electrical engineering and electronics.
Networking:
Professional Associations: Joining organizations like the Institute of Electrical and Electronics Engineers (IEEE) can provide networking opportunities, resources, and continued education.
Online Forums and Groups: Websites like Stack Exchange (for electrical engineering), online forums, and even Reddit have active communities where you can ask questions, share knowledge, and learn from others.
Practice and Experimentation:
Regularly challenge yourself with new projects or problems to solve. The more you apply your knowledge, the more intuitive and deeply ingrained it will become.
Stay Updated:
Industry Trends: Electrics, like all fields of technology, is continuously evolving. Ensure you're aware of the latest tools, technologies, and methodologies in the field.
Regular Refreshers: Periodically revisit foundational concepts to ensure your foundational knowledge remains strong.
Soft Skills:
While technical knowledge is crucial, don't underestimate the importance of communication, teamwork, problem-solving, and critical thinking skills. These are often essential, especially if you aim to take on leadership or interdisciplinary roles.
Teach and Mentor:
Once you've gained a certain level of expertise, consider teaching or mentoring others. This not only benefits the community but can also reinforce and expand your understanding.
Becoming an electrician requires dedication, passion, and a willingness to continually push your boundaries. With focused effort and the right resources, you can achieve a high level of expertise and enjoy a fulfilling career in the world of electronics.
}
\Remarks{Remarks}{
The rejected response, while containing much unnecessary and potentially irrelevant information (e.g., suggesting an electrical engineering degree), uses extensive formatting (numbered lists with sub-points). A reward model overly sensitive to formattedness might prefer it over the more concise and directly relevant chosen response.
}
