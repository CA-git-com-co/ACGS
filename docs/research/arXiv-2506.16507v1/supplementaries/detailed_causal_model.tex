\section{Causal Model and Augmentation Details}
\label{sec:causal_model_details}

This appendix provides further details on the causal framework underpinning \carma{} and discusses various data augmentation strategies in the context of robust reward modeling.

\subsection{Elaboration on the Causal Model}
The causal graph presented in Figure \ref{fig:causal_graph} (Section \ref{subsec:causal_graph}) models the generation of an answer $\mathrm{A}$ and the formation of its attributes. The query $\mathrm{Q}$ influences the generator's latent \textit{intent} $\mathcal{I}$. This intent, along with unobserved generator-specific confounders $\mathcal{U}$ (e.g., inherent stylistic preferences, verbosity tendencies, pre-existing biases), leads to the textual answer $\mathrm{A}$. The answer $\mathrm{A}$ then manifests both \textit{causal attributes} $\mathrm{C}(\mathrm{A})$ (e.g., factuality, relevance) and \textit{spurious attributes} $\mathrm{SP}(\mathrm{A})$ (e.g., length, specific formatting, politeness). The true, idealized reward $\mathrm{R}^*$ is assumed to be a function only of $\mathrm{Q}$ and $\mathrm{C}(\mathrm{A})$.

The challenge in training a reward model $\hat{\mathrm{R}}_\theta$ arises because $\mathrm{SP}(\mathrm{A})$ can become correlated with $\mathrm{R}^*$ in the training data. This correlation can occur if $\mathcal{U}$ influences both the choice of spurious features and the aspects that contribute to causal quality, or simply because certain spurious features happen to co-occur with preferred answers in $\mathcal{D}_{\mathrm{pref}}$. Without explicit guidance, $\hat{\mathrm{R}}_\theta$ may learn to rely on these spurious correlations, leading to reward hacking. \carma{}'s data augmentation strategy aims to provide this explicit guidance by generating new answer pairs that help $\hat{\mathrm{R}}_\theta$ disentangle $\mathrm{C}(\mathrm{A})$ from $\mathrm{SP}(\mathrm{A})$.

\subsection{\carma{}'s Causal Augmentation: Attribute Isolation}
\label{app:carma_causal_aug}
\carma{}'s primary strategy for enhancing sensitivity to causal attributes involves \textit{Attribute Upgradation/Degradation}. This generates pairs $(\tilde{\mathrm{A}}^{(C_j \leftarrow \text{upgraded/degraded})}, \mathrm{A})$ or $(\mathrm{A}, \tilde{\mathrm{A}}^{(C_j \leftarrow \text{upgraded/degraded})})$ by prompting an LLM to modify an original answer $\mathrm{A}$ (from $\mathcal{D}_{\mathrm{pref}}$) along a single causal attribute $C_j$ while attempting to keep other attributes constant. This provides a targeted signal about the marginal contribution of $C_j$.

\subsubsection{Comparison with Relevance Contrast Augmentation}
An alternative strategy, \textit{Relevance Contrast Augmentation} (used in RRM-style approaches \citep{liu2024rrm}, termed ``non-contextuals'' therein), involves pairing a relevant answer $\mathrm{A}_1$ (for query $\mathrm{Q}$) with an irrelevant answer $\mathrm{B}_2$ (e.g., an answer to a different query, so $C(B_2 \mid \mathrm{Q}) \approx \mathbf{0}$), labeled $\mathrm{A}_1 \succ \mathrm{B}_2$.

While Relevance Contrast establishes a baseline understanding of relevance, \carma{}'s Attribute Isolation offers:
\begin{itemize}[itemsep=0pt,left=8pt]
    \item \textbf{Specificity and Nuance:} It directly teaches about individual causal attributes ($C_j$), enabling the RM to learn a compositional understanding of quality and distinguish between relevant answers differing subtly in one dimension.
    \item \textbf{Data Efficiency for Complex Attributes:} Focusing changes along one attribute creates diverse, targeted examples for each quality facet.
\end{itemize}
\carma{}'s attribute-specific counterfactuals thus provide a richer, more disentangled signal than broad relevance contrasts alone.

\subsection{Neutral Augmentation Strategies}
Neutral augmentations aim to make the reward model invariant to spurious attributes when causal content is held constant or is irrelevant.

\subsubsection{Common Spurious Perturbation Methods (Not a primary \carma{} strategy)}
Several methods focus on general spurious perturbations:
\paragraph{1. Direct Spurious Feature Perturbation (e.g., Paraphrasing, Formatting Changes):}
This involves taking an answer $\mathrm{A}$ and generating $\tilde{\mathrm{A}}^{(SP \leftarrow sp')}$ by applying meaning-preserving transformations (e.g., paraphrasing) intended to alter only $\mathrm{SP}(\mathrm{A})$ while preserving $\mathrm{C}(\mathrm{A})$. The pair $(\mathrm{A}, \tilde{\mathrm{A}}^{(SP \leftarrow sp')})$ is labeled as a tie. This is central to benchmarks like reWordBench \citep{wu2025rewordbench}.

\paragraph{2. Rewrites of Rewrites (e.g., RATE \citep{reber2024rate}):}
RATE uses sequential rewrites for robust causal effect estimation. Adapted for augmentation, multiple causally-equivalent rewrites of an answer could form neutral pairs.

\textit{Challenges with these General Methods:}
\begin{itemize}[itemsep=0pt,left=8pt]
    \item \textbf{Unknown/Unspecified Spurious Features:} It's hard to a priori identify and target all spurious features an RM might exploit.
    \item \textbf{Preserving Causal Content:} Ensuring "spurious" perturbations don't inadvertently alter causal meaning is difficult.
\end{itemize}

\subsubsection{Neutral Augmentation Strategies developed in this work}
We developed the following two strategies for neutral augmentation.

\paragraph{1. Irrelevant Query Neutrals (IQN):}
\carma{} generates these neutral pairs efficiently by leveraging its existing pool of answers (original or causally augmented). Given two answers, $\mathrm{B}_1$ and $\mathrm{B}_2$, that were generated or selected for a specific query $\mathrm{Q}_{\text{orig}}$, \carma{} creates a neutral pair by associating them with a \textit{new, unrelated query} $\mathrm{Q}_{\text{irrelevant}}$.
For this $\mathrm{Q}_{\text{irrelevant}}$, both $\mathrm{B}_1$ and $\mathrm{B}_2$ are now contextually irrelevant; their causal attribute scores $C(B_1 | \mathrm{Q}_{\text{irrelevant}})$ and $C(B_2 | \mathrm{Q}_{\text{irrelevant}})$ are effectively zero (or very low). Despite potentially different spurious attributes $SP(B_1)$ and $SP(B_2)$, the pair $(\mathrm{B}_1, \mathrm{B}_2)$ is presented to the reward model with query $\mathrm{Q}_{\text{irrelevant}}$ and labeled as a tie.
This teaches the RM that when answers are equally and maximally irrelevant to the current query, their differing spurious features should not induce a preference.

\paragraph{2. Causally-Aligned Neutrals (CAN):}
This method directly leverages the original preference pairs or the outputs of causal augmentation.
\begin{itemize}[itemsep=0pt,left=8pt]
    \item Given an original preference pair from $\mathcal{D}_{\mathrm{pref}}$, say $(\mathrm{A}_1, \mathrm{A}_2)$ where $\mathrm{A}_1 \succ \mathrm{A}_2$, we generate $\tilde{\mathrm{A}}_{2}^{(C \leftarrow C(A_1))}$ by rewriting $\mathrm{A}_2$ to match the causal attribute profile of $\mathrm{A}_1$, while instructing the LLM to retain the spurious characteristics $SP(A_2)$ of the original $\mathrm{A}_2$. The pair $(\mathrm{A}_1, \tilde{\mathrm{A}}_{2}^{(C \leftarrow C(A_1))})$ is then labeled as a tie. A symmetric pair can also be generated.
    \item Similarly, if we have an answer $\mathrm{A}$ and its causally degraded version $\tilde{\mathrm{A}}^{(C_j \leftarrow \text{degraded})}$ (from $\mathcal{D}_{\mathrm{causal}}$), we can attempt to reconstruct the degraded version by prompting an LLM to restore $C_j$ to its state in $\mathrm{A}$, while aiming to preserve the spurious features of $\tilde{\mathrm{A}}^{(C_j \leftarrow \text{degraded})}$. If successful, this reconstructed version, $\tilde{\mathrm{A}}'_{\text{reconstr}}$, would form a neutral pair $(\mathrm{A}, \tilde{\mathrm{A}}'_{\text{reconstr}})$ labeled as a tie.
\end{itemize}
The core idea is to teach invariance to the spurious differences that remain \textit{after} causal attributes have been aligned or restored.
Moreover, applying CAN to counterfactually generated data from $\mathcal{D}_{\mathrm{causal}}$ helps mitigate imperfections in oracle rewrites—an issue highlighted in the RATE paper \citep{reber2024rate}, which notes that LLM edits often unintentionally modify "off-target attributes" (e.g., introducing formality, removing HTML tags). CAN thereby enhances robustness on two fronts: (1) disentangling spurious correlations in original data, and (2)  neutralizing new biases introduced during causal augmentation.
This helps in enhancing model's robustness against confounding signals in the data. While this method is sound theoretically, we qualitatively find that the approximation of $C(A_w)$ by $C(\tilde{A}_l)$ is not perfect. Furthermore, some spurious attributes $SP'(\tilde{A}_l) \subset SP(\tilde{A}_l)$ vary when we move causal attributes. Invariance to these attributes $SP'(\tilde{A}_l)$ is not captured by CAN. For these reasons, we encourage future work for improving this neutral augmentation strategy.