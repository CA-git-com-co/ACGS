\usepackage[utf8]{inputenc}
\usepackage{verbatim}
\usepackage{amsfonts}  % Used for \mathbb.
\usepackage{amssymb}
\usepackage{physics}  % Used for \dd.
\usepackage{mathtools}  % Used for dcases environment.

% COLOR
% \usepackage[usenames,dvipsnames,table]{xcolor} moved to main.tex

% FIGURES
\usepackage{adjustbox}
\usepackage{graphicx}
\usepackage[labelfont=bf]{caption}
\usepackage[format=hang]{subcaption}
\usepackage{pstricks, pst-node}
\usepackage{floatrow}
\usepackage{wrapfig}

% TABLES
\usepackage{booktabs}
\usepackage{multirow}
\usepackage{float}
\usepackage{colortbl}
\usepackage{makecell}

% BIBLIOGRAPHY
\usepackage[numbers,sort&compress]{natbib}
\newcommand\authoryearcite[1]{%
  (\citeauthor{#1}, \citeyear{#1})}
\newcommand\authoryearcitet[1]{%
  \citeauthor{#1} (\citeyear{#1})}
\newcommand\authoryearcitetwo[2]{%
  (\citeauthor{#1}, \citeyear{#1}; \citeauthor{#2}, \citeyear{#2})}

% HYPERREF setup
\usepackage[all]{hypcap}
\hypersetup{colorlinks}
\hypersetup{linktoc=all}
\hypersetup{citecolor=Violet}
\hypersetup{linkcolor=black}
\hypersetup{urlcolor=MidnightBlue}

% CLEVEREF (must come after HYPERREF)
\usepackage[nameinlink]{cleveref}
\creflabelformat{equation}{#1#2#3}
\Crefname{equation}{Eq.}{Eqs.}
\Crefname{assumption}{Assumption}{Assumptions}
\Crefname{prop}{Proposition}{Propositions}
\Crefname{lem}{Lemma}{Lemmas}

% ALGORITHMS
\usepackage{algorithm}
\usepackage[noend]{algpseudocode}

\definecolor{darkgreen}{rgb}{0,0.5,0}
\definecolor{darkred}{rgb}{0.5,0,0}
\newcommand{\DEF}[1]{\item[\textcolor{darkgreen}{\textbf{def}}] #1}
\newcommand{\inner}[2]{{#1}^{\!\top}#2}
\newcommand{\R}[0]{\mathds{R}}
\newcommand{\COMMENTLINE}[1]{\vspace{3pt}\item[\textcolor{darkred}{\em \#}]\textcolor{darkred}{\em #1}}

% PYTHON CODE
% (1) First compile with this:
% \usepackage{minted}
% (2) Then recompile with this:
% \usepackage[finalizecache,cachedir=.]{minted}
% (3) Submit to arXiv with this:
\usepackage[frozencache,cachedir=.]{minted}
\setminted{
  fontsize=\footnotesize,
  breaklines=true,
  escapeinside=|| % Specify the escape characters
}
\newcommand\pythoncode[1]{\small\inputminted[
    frame=lines,
    breaklines,
    breakafter=d,
    fontsize=\ttfamily\scriptsize,
    % baselinestretch=1.0,
    % style=colorful,
    linenos,
    escapeinside=||]{python}{#1}}

% THEOREMS
\usepackage{amsthm}
\theoremstyle{definition}
\newtheorem{definition}{Definition}
\theoremstyle{remark}
\newtheorem{remark}{Remark}
\theoremstyle{plain}
\newtheorem{proposition}{Proposition}
\newtheorem{lemma}{Lemma}
\newtheorem{theorem}{Theorem}

%%% HELPER CODE FOR DEALING WITH EXTERNAL REFERENCES
\usepackage{xr}
\makeatletter
\newcommand*{\addFileDependency}[1]{
  \typeout{(#1)}
  \@addtofilelist{#1}
  \IfFileExists{#1}{}{\typeout{No file #1.}}
}
\makeatother
\newcommand*{\myexternaldocument}[1]{
    \externaldocument{#1}
    \addFileDependency{#1.tex}
    \addFileDependency{#1.aux}
}
%%% END HELPER CODE

% NAMES
\usepackage{xspace}
\newcommand{\ResultsColabURL}{https://colab.research.google.com/github/google-deepmind/alphaevolve\_results/blob/master/mathematical\_results.ipynb}
\newcommand{\ResultsColab}{the accompanying \href{\ResultsColabURL}{Google Colab}}
\newcommand{\method}{\emph{AlphaEvolve}\xspace}
\newcommand{\CentralLogger}{ProgramsDB}
\newcommand{\CentralLoggerColor}[1]{\textcolor{red}{#1}}
\newcommand{\CentralLoggerDiagram}{\CentralLoggerColor{\CentralLogger}}
\newcommand{\Sampler}{Sampler}
\newcommand{\SamplerColor}[1]{\textcolor{blue}{#1}}
\newcommand{\SamplerDiagram}{\SamplerColor{\Sampler}}
\newcommand{\Summarizer}{Evaluator}
\newcommand{\SummarizerColor}[1]{\textcolor{OliveGreen}{#1}}
\newcommand{\SummarizerDiagram}{\SummarizerColor{\Summarizer}}

\newcommand\blfootnote[1]{%
  \begingroup
  \renewcommand\thefootnote{}\footnote{#1}%
  \addtocounter{footnote}{-1}%
  \endgroup
}

\providecommand{\assumptionname}{Assumption}
\providecommand{\lemmaname}{Lemma}
\providecommand{\propositionname}{Proposition}
\providecommand{\theoremname}{Theorem}

\usepackage[normalem]{ulem}
\newcommand{\cancel}{\sout}

% Configure listings for Python with diff highlighting
\usepackage{listings}

\definecolor{diffgreen}{rgb}{0.0, 0.6, 0.0}
\definecolor{diffred}{rgb}{0.8, 0.0, 0.0}
\definecolor{diffhead}{rgb}{0.5, 0.5, 0.5}
\definecolor{codegray}{rgb}{0.5,0.5,0.5}
\definecolor{codeblue}{rgb}{0.0,0.0,0.6}
\definecolor{codepurple}{rgb}{0.58,0,0.82}
\definecolor{codeorange}{rgb}{1.0, 0.5, 0.0}
\definecolor{backcolour}{rgb}{0.97,0.97,0.97}
\definecolor{highlightblue}{rgb}{0.9, 0.9, 1.0}
\definecolor{highlightyellow}{rgb}{0.95, 0.95, 0.8}
\definecolor{highlightorange}{rgb}{1.0, 0.95, 0.8}

\setlength{\fboxsep}{0pt}

\newcommand{\myhighlightblue}[1]{\colorbox{highlightblue}{\strut\mbox{#1}\hspace*{\fill}}}
\newcommand{\myhighlightyellow}[1]{\colorbox{highlightyellow}{\strut\mbox{#1}\hspace*{\fill}}}
\newcommand{\myhighlightorange}[1]{\colorbox{highlightorange}{\strut\mbox{#1}\hspace*{\fill}}}

% Configure listings for Python with diff highlighting
\lstdefinestyle{pydiff}{
    language=Python,
    % backgroundcolor=\color{backcolour},
    commentstyle=\color{codepurple}\itshape,
    keywordstyle=\color{codeblue}\bfseries,
    numberstyle=\tiny\color{codegray}, % Line numbers remain tiny
    stringstyle=\color{codeorange},
    basicstyle=\ttfamily\scriptsize,
    % -----------------------------
    breakatwhitespace=false,
    breaklines=true,
    captionpos=b,
    keepspaces=true,
    numbers=left,
    numbersep=5pt,
    showspaces=false,
    showstringspaces=false,
    showtabs=false,
    tabsize=2,
    frame=single, % Add a frame around the code
    framerule=0.4pt,
    rulecolor=\color{lightgray},
    % --- Custom Diff Highlighting ---
    morecomment=[f][\color{diffgreen}]{+}, % Added lines start with +
    morecomment=[f][\color{diffred}]{-},   % Removed lines start with -
    morecomment=[f][\color{diffhead}]{@},  % Header lines start with @
    morecomment=[f][\color{codeorange}]{...},  % ... lines
    moredelim=[is][\myhighlightblue]{^1}{^}, % Use ^ as invisible marker
    moredelim=[is][\myhighlightyellow]{^2}{^}, % Use ^ as invisible marker
    moredelim=[is][\myhighlightorange]{^3}{^}, % Use ^ as invisible marker
    % --------------------------------
    xleftmargin=0pt,
    xrightmargin=0pt,
}

\lstdefinestyle{pydiffmedium}{
    language=Python,
    % backgroundcolor=\color{backcolour},
    commentstyle=\color{codepurple}\itshape,
    keywordstyle=\color{codeblue}\bfseries,
    numberstyle=\tiny\scalefont{0.69}\color{codegray}, % Line numbers remain tiny
    stringstyle=\color{codeorange},
    basicstyle=\ttfamily\scriptsize\scalefont{0.69},
    % -----------------------------
    breakatwhitespace=false,
    breaklines=true,
    captionpos=b,
    keepspaces=true,
    numbers=left,
    numbersep=5pt,
    showspaces=false,
    showstringspaces=false,
    showtabs=false,
    tabsize=2,
    frame=single, % Add a frame around the code
    framerule=0.4pt,
    rulecolor=\color{lightgray},
    % --- Custom Diff Highlighting ---
    morecomment=[f][\color{diffgreen}]{+}, % Added lines start with +
    morecomment=[f][\color{diffred}]{-},   % Removed lines start with -
    morecomment=[f][\color{diffhead}]{@},  % Header lines start with @
    morecomment=[f][\color{codeorange}]{...},  % ... lines
    moredelim=[is][\myhighlightblue]{^1}{^}, % Use ^ as invisible marker
    moredelim=[is][\myhighlightyellow]{^2}{^}, % Use ^ as invisible marker
    moredelim=[is][\myhighlightorange]{^3}{^}, % Use ^ as invisible marker
    % --------------------------------
    xleftmargin=0pt,
    xrightmargin=0pt,
}
